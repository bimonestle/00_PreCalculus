\documentclass{article}

\usepackage{amsmath}
\usepackage{array}
\usepackage{geometry}
\usepackage{enumitem}
\usepackage{cancel}
\usepackage{setspace}

% Setting specific margins
\geometry{left=1.5in, top=1in, right=1.5in, bottom=1in}

% Setting paper size
\geometry{a4paper}  % or letterpaper, a5paper, etc.

% Increasing the vertical padding
\renewcommand{\arraystretch}{1.6} % Adjust the multiplier as needed

\title{CHAPTER 0\\
REVIEW OF ALGEBRA\\
03. Properties of Exponents and Radicals
}
\author{Exercised by: Rizal Bimanto}
\date{}

\begin{document}
\maketitle

\section{Summary}\par
\begin{onehalfspace}
    The product $x \cdot x \cdot x$ of 3 $x$'s is abbreviated $x^{3}$.
    In general, for $n$ a positive integer, $x^{n}$ is the abbreviation product
    of $n x$'s. The letter $n$ in $x^{n}$ is called the \textbf{the exponent},
    and $x$ is called the \textbf{base}. More specifically, if $n$ is positive integers we have:

    \begin{enumerate}
        \item $x^n = \underbrace{x \cdot x \cdot \ldots \cdot x} _ {n \text{ factors}}$
        \item $\frac {1}{\underbrace{x \cdot x \cdot x \cdot \ldots \cdot x}_{n \text{ factors}}}$ for $x \neq 0$
        \item $\frac {1}{x^{-n}} = x^{n}$ for $x \neq 0$
        \item $x^{0} = 1$
    \end{enumerate}

    If $r^{n} = x$, where $n$ is a positive integer, then $r$ is an $n$th \textbf{root} of $x$.
    Second roots, the case $n = 2$, are called \textbf{squared roots};
    and third roots, the case $n = 3$, are called \textbf{cube roots}.

    Some numbers do not have an $n$th root that is a real number.
    For example, since the square of any real number is non-negative:
    there is no real number that is a square root of $-4$.

    The principal of $n$th root of $x$ is the $n$th root of $x$
    that is positive if $x$ is positive and is negative if $x$ is negative
    and $n$ is odd. We denote the principal $n$th root of
    $x$ by $\sqrt[n]{x}$:

    \begin{center}
    \[
    \sqrt[n]{x} \text{ is }
        \begin{cases} 
        \text{positive if } x \text{ is positive} \\
        \text{negative if } x \text{ is negative and } n \text{ is odd}
        \end{cases}
    \]
    \end{center}

    For example, $\sqrt[2]{9} = 3$, $\sqrt[3]{-8} = -2$ and $\sqrt[3]{\frac{1}{27}} = \frac{1}{3}$
    We define $\sqrt[n]{0} = 0$.
    \newline
    The symbol of $\sqrt[n]{x}$ is called radical.

    Here are the basic laws of exponent and radicals:
    \bigskip
    \bigskip

    \begin{center}
        \textbf{Law} \hspace{3cm} \textbf{Example(s)}
        \newline
        \begin{tabular}{| >{$}l<{$} | >{$}l<{$} |c|c|}
        \hline
        1.\ x^m \cdot x^n = x^{m+n} & 2^3 \cdot 2^5 = 2^{8} = 256; \quad x^2 \cdot x^3 = x^5 \\
        2.\ x^0 = 1 & 2^0 = 1 \\
        3.\ x^{-n} = \frac{1}{x^n} & 2^{-3} = \frac{1}{2^3} = \frac{1}{8} \\
        4.\ \frac{1}{x^{-n}} = x^{n} & \frac{1}{2^{-3}} = 2^{3} = 8; \frac{1}{x^{-5}} = x^{5}\\
        5.\ \frac{x^{m}}{x^{n}} = x^{m-n} = \frac{1}{x^{n-m}} & \frac{2^{12}}{2^{8}} = 2^{4} = 16; \frac{x^{8}}{x^{12}} = x^{-4} = \frac{1}{x^{4}}\\
        6.\ \frac{x^{m}}{x^{m}} = 1 & \frac{2^{4}}{2^{4}} = 1 \\
        7.\ \left(x^{m}\right)^{n} = x^{mn} & \left(2^{3}\right)^{5} = 2^{15};\left(x^{2}\right)^{3} = x^{6}\\
        8.\ \left(xy^{n}\right) = x^{n}y^{n} & \left(2\cdot4\right)^{3} = 2^{3}\cdot4^{3} = 8 \cdot 64 = 512\\
        9.\ \left(\frac{x}{y}\right)^{n} = \frac{x^{n}}{y^{n}} & \left(\frac{2}{3}\right)^{3} = \frac{2^{3}}{3^{3}} = \frac{8}{27}\\
        10.\ \left(\frac{x}{y}\right)^{-n} = \left(\frac{y}{x}\right)^{n} & \left(\frac{3}{4}\right)^{-2} = \left(\frac{4}{3}\right)^{2} = \frac{16}{9}\\
        11.\ x^{\frac{1}{n}} = \sqrt[n]{x^{1}} & 3^{\frac{1}{5}} = \sqrt[5]{3^{1}}\\
        12.\ x^{\frac{-1}{n}} = \frac{1}{x^{\frac{1}{n}}} = \frac{1}{\sqrt[n]{x}} & 4^{\frac{-1}{2}} = \frac{1}{4^{\frac{1}{2}}} = \frac{1}{\sqrt[2]{4}} = \frac{1}{2}\\
        13.\ \sqrt[n]{x} \sqrt[n]{y} = \sqrt[n]{xy} & \sqrt[3]{9} \sqrt[3]{2} = \sqrt[3]{9\cdot2} = \sqrt[3]{18}\\
        14.\ \frac{\sqrt[n]{x}}{\sqrt[n]{y}} = \sqrt[n]{\frac{x}{y}} & \frac{\sqrt[3]{90}}{\sqrt[3]{10}} = \sqrt[3]{\frac{90}{10}} = \sqrt[3]{9}\\
        15.\ \sqrt[m]{\sqrt[n]{x}} = \sqrt[mn]{x} & \sqrt[3]{\sqrt[4]{2}} = \sqrt[3 \cdot 4]{2} = \sqrt[12]{2}\\
        16.\ x^{\frac{m}{n}} = \sqrt[n]{x^{m}} = \left( \sqrt[n]{x} \right)^{m} & 8^{\frac{2}{3}} = \sqrt[3]{8^{2}} = \left( \sqrt[3]{8} \right)^{2} = 2^{2} = 4\\
        17.\ \left( \sqrt[m]{x^{m}} \right) = x & \left( \sqrt[8]{7} \right)^{8} = 7 \\
        \hline
        \end{tabular}
    \end{center}

\end{onehalfspace}

\section{Problems 0.3}\par
\textit{In Problems 1 - 14, simplify and express all answers in terms of positive exponent}

\begin{onehalfspace}
    \begin{enumerate}
        \item $\left( 2^{3} \right)$ $\left( 2^{2} \right)$
        \begin{itemize}
            \item $2^{(3 + 2)}$
            \item $2^{5}$
            \item $32$
        \end{itemize}
        

        \item $x^{6}x^{9}$
        \begin{itemize}
            \item $x^{(6 + 9)}$
            \item $x^{(15)}$
        \end{itemize}
        

        \item $17^{5} \cdot 17^{2}$
        \begin{itemize}
            \item $17^{(5 + 2)}$
            \item $17^{(7)}$
            \item $410,338,673$
        \end{itemize}
        

        \item $z^{3}zz^{2}$
        \begin{itemize}
            \item $z^{(3 + 1 + 2)}$
            \item $z^{6}$
        \end{itemize}
        

        \item $\frac{x^{3}x^{5}}{y^{9}y^{5}}$
        \begin{itemize}
            \item $\frac{x^{(3 + 5)}}{y^{(9 + 5)}}$
            \item $\frac{x^{8}}{y^{14}}$
        \end{itemize}
        

        \item $\left( x^{12} \right)^{4}$
        \begin{itemize}
            \item $x^{(12 \cdot 4)}$
            \item $x^{48}$
        \end{itemize}
        

        \item $\frac{\left( a^{3} \right)^{7}}{\left( b^{4} \right)^{5}}$
        \begin{itemize}
            \item $\frac{a^{3 \cdot 7}}{b^{4 \cdot 5}}$
            \item $\frac{a^{21}}{b^{20}}$
        \end{itemize}
        

        \item $\left( \frac{13^{14}}{13} \right)^{2}$
        \begin{itemize}
            \item $\frac{13^{(14 \cdot 2)}}{13^{2}}$
            \item $\frac{13^{28}}{13^{2}}$
            \item $13^{(28 - 2)}$
            \item $13^{26}$
        \end{itemize}
        

        \item $\left( 2x^{2}y^{3} \right)^{3}$
        \begin{itemize}
            \item $2^{3}x^{(2 \cdot 3)}y^{(3 \cdot 3)}$
            \item $8x^{6}y^{9}$
        \end{itemize}
        

        \item $\left( \frac{w^{2}s^{3}}{y^{2}} \right)^{2}$
        \begin{itemize}
            \item $\frac{w^{(2 \cdot 2)}s^{(3 \cdot 2)}}{y^{(2 \cdot 2)}}$
            \item $\frac{w^{4}s^{6}}{y^{4}}$
        \end{itemize}
        
        
        \item $\left( \frac{x^{9}}{x^{5}} \right)$
        \begin{itemize}
            \item $x^{(9 - 5)}$
            \item $x^{4}$
        \end{itemize}
        

        \item $\left( \frac{2a^{4}}{7b^{5}} \right)^{6}$
        \begin{itemize}
            \item $\frac{2^{6}a^{(4 \cdot 6)}}{7^{6}b^{(5 \cdot 6)}}$
            \item $\frac{2^{6}a^{24}}{7^{6}b^{30}}$
        \end{itemize}

        \item $\frac{(y^{3})^{4}}{(y^{2})^{3}y^{2}}$
        \begin{itemize}
            \item $\frac{y^{3 \cdot 4}}{y^{2 \cdot 3 + 2}}$
            \item $\frac{y^{12}}{y^{8}}$
            \item $y^{(12 - 8)}$
            \item $y^{4}$
        \end{itemize}
        
        \item $\frac{(x^{2})^{3}(x^{3})^{2}}{(x^{3})^{4}}$
        \begin{itemize}
            \item $\frac{x^{(2 \cdot 3)x^{(3 \cdot 2)}}}{x^{(3 \cdot 4)}}$
            \item $\frac{x^{6}x^{6}}{x^{12}}$
            \item $\frac{x^{(6 + 6)}}{x^{12}}$
            \item $\frac{x^{12}}{x^{12}}$
            \item $x^{(12 - 12)}$
            \item $x^{0}$
            \item $1$
        \end{itemize}
    \end{enumerate}
\end{onehalfspace}

\end{document}