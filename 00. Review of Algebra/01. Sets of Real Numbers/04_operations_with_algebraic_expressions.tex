\documentclass{article}

\usepackage{amsmath}
\usepackage{array}
\usepackage{geometry}
\usepackage{enumitem}
\usepackage{cancel}
\usepackage{setspace}

% Setting specific margins
\geometry{left=1.5in, top=1in, right=1.5in, bottom=1in}

% Setting paper size
\geometry{a4paper}  % or letterpaper, a5paper, etc.

% Increasing the vertical padding
\renewcommand{\arraystretch}{1.6} % Adjust the multiplier as needed

\title{CHAPTER 0\\
REVIEW OF ALGEBRA\\
04. Operations with Algebraic Expressions
}
\author{Exercised by: Rizal Bimanto}
\date{}

\begin{document}
\maketitle

\begin{onehalfspace}
    \section{Summary}\par

    Algebraic expressions with exactly 1 term are
    called \textbf{monomials}.
    \begin{center}
        $3x^{2}$
    \end{center}\par

    Algebraic expressions with exactly 2 terms are
    called \textbf{binomials}.
    \begin{center}
        $3x^{2} + 3x$
    \end{center}
    
    Algebraic expressions with exactly 3 terms are
    called \textbf{trinomials}.
    \begin{center}
        $3x^{2} + 3x + z$
    \end{center}

    Algebraic expressions with more terms are
    called \textbf{polynomials}.
    \begin{center}
        $3x^{2} + 3x + z + 6 + b^{3}$
    \end{center}
    \par

    \textbf{Special Products}
    \begin{enumerate}
        \item $x(y + z) = xy + xz$
        \item $(x + a)(x + b) = x^{2} + x(a + b) + ab$
        \item $(ax + c)(bx + d) = abx^{2} + x(ad + bc) + cd$
        \item $(x + a)^{2} = x^{2} + 2ax + a^{2}$
        \item $(x - a)^{2} = x^{2} -2ax + a^{2}$
        \item $(x + a)(x - a) = x^{2} - a^{2}$
        \item $(x + a)^{3} = x^{3} + 3ax^{2} + 3a^{2}x + a^{3}$
        \item $(x - a)^{3} = x^{3} - 3ax^{2} + 3a^{2}x - a^{3}$
    \end{enumerate}

    \section{Long Division}

    Divide $2x^{3} - 14x - 5$ by $x - 3$

\[
\begin{array}{r|l}
    & 2x^2 + 6x + 4 \ \leftarrow Quotient\\
    \cline{2-2}
    Divisor \rightarrow  (x - 3 \big) & 2x^3 + 0x^2 - 14x - 5 \\
    & -(2x^3 - 6x^2) \\
    \cline{2-2}
    & \phantom{-}6x^2 - 14x \\
    & -(6x^2 - 18x) \\
    \cline{2-2}
    & \phantom{--}4x - 5 \\
    & -(4x - 12) \\
    \cline{2-2}
    & \phantom{---}7 \ \leftarrow Remainder
\end{array}
\]

So the result of $2x^{3} - 14x - 5$ by $x - 3$ is \par
\begin{center}
    $2x^{2} + 6x + 4 + \frac{7}{x - 3}$
\end{center}

\begin{itemize}
    \item $\frac{Dividend}{Divisor} = Quotient + \frac{Remainder}{Divisor}$
\end{itemize}

A way of checking a division is to verify that \par

\begin{itemize}
    \item $Dividend = \left( Quotient + \frac{Remainder}{Divisor} \right) Divisor$ \par
    \item $Dividend = Quotient \cdot Divisor + \frac{Remainder}{Divisor} \cdot Divisor$ \par
    \item $Dividend = Quotient \cdot Divisor + \frac{Remainder}{\cancel{Divisor}} \cdot \cancel{Divisor}$ \par
    \item $Dividend = Quotient \cdot Divisor + Remainder$\par
\end{itemize}

By using this equation, you should be able to verify the result of the example.
\end{onehalfspace}
\end{document}