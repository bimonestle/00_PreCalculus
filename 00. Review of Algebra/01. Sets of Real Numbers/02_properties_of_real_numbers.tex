\documentclass{article}

\usepackage{amsmath}
\usepackage{geometry}
\usepackage{enumitem}

% Setting specific margins
\geometry{left=1.5in, top=1in, right=1.5in, bottom=1in}

% Setting paper size
\geometry{a4paper}  % or letterpaper, a5paper, etc.

\title{CHAPTER 0\\
REVIEW OF ALGEBRA\\ 
02. Properties of Real Numbers
}
\author{Exercised by: Rizal Bimanto}
\date{}

\begin{document}
\maketitle

A list of properties of the real numbers.

\begin{enumerate}
    \item \textbf{The Transitive Property of Equality}\par
    \begin{center}
        If $a = b$ and $b = c$, then $a = c$
    \end{center}
    
    \item \textbf{The Closures Properties of Addition and Multiplication}\par
    For all real numbers $a$ and $b$,
    there are unique real numbers $a + b$ and $ab$

    \item \textbf{The Commutative Properties of Addition and Multiplication}\par
    \begin{center}
        $a + b = b + a$ and $ab = ba$
    \end{center}

    \item \textbf{The Associative Properties of Addition and Multiplication}\par
    \begin{center}
        $a + (b + c) = (a + b) + c$ and $a(bc) = (ab)c$
    \end{center}

    \item \textbf{The Identity Properties}\par
    There are unique real numbers denoted 0 and 1 such that, for each real number $a$,\par
    \begin{center}
        $0 + a = a$ and $1a = a$
    \end{center}

    \item \textbf{The Inverse Properties}\par
    For each real number $a$, there is unique real number denoted $-a$ such that
    \begin{center}
        $a + (-a) = 0$
    \end{center}
    The number $-a$ is called the \textbf{negative} of a.\par
    For each real number $a, except$ 0, there is a unique real number denoted $a^{-1}$ such that
    \begin{center}
        $a \times a^{-1} = 1$
    \end{center}
    The number $a^{-1}$ is called the \textbf{reciprocal} of $a$

    \item \textbf{The Distributive Properties}\par
    \begin{center}
        $a(b + c) = ab + ac$ and $(b + c)a = ba + ca$\par
        $0 \times a = 0 = a \times 0$
    \end{center}
\end{enumerate}
\break

\section{Problems 0.2}\par
\textit{In Problems 1 - 10, determine the truth of each statement}

\begin{enumerate}
    \item Every real number has a reciprocal. \par
    \textit{False. Except 0}

    \item The reciprocal of $6.6$ is $0.1515\dots$ \par
    \textit{$\frac{1}{6.6} = 0.1515\dots$}. True

    \item The negative of 7 is $\frac{-1}{7}$ \par
    \textit{-(7) = -7. False. It should be -7}

    \item $1(x \times y) = (1 \times x)(1 \times y)$ \par
    \textit{True. It can be simplified as $xy$}

    \item $-x + y = -y + x$\par
    \textit{False. $-x + y = y - x$}

    \item $(x + 2)(4) = 4x + 8$\par
    \textit{True.}

    \item $\frac{x + 3}{5} = \frac{x}{5} + 3$\par
    \textit{False. $\frac{x + 3}{5} = \frac{x}{5} + \frac{3}{5}$}

    \item $3 \left( \frac{x}{4} \right) = \frac{3x}{4}$\par
    \textit{True.}

    \item $2(x \times y) = (2x) \times (2y)$\par
    \textit{False. $2(x \times y) = (2x) \times (2y) = 2xy$}

    \item $x(4y) = 4xy$\par
    \textit{True.}\par
\end{enumerate}

\textit{In Problems 11-20, state which properties of the real numbers are being used.}
\begin{enumerate}[start=11]
    \item $2(x+y) = 2x + 2y$\par
    \textit{The Distributive Properties}

    \item $(x + 5.2) + 0.7y = x + (5.2 + 0.7y)$\par
    \textit{The Associative Property of Addition}
    
    \item $2(3y) = (2 \cdot 3)y$\par
    \textit{The Associative Property of Multiplication}    

    \item $\frac{a}{b} = \frac{1}{b} \cdot a$\par
    \textit{The Inverse Property}
    
    \item $5(b - a) = (a - b)(-5)$\par
    \textit{The Commutative Property of Multiplication and Distributive}
    
    \item $y + (x + y) = (y + x) + y$\par
    \textit{The Commutative Property of Addition}
    
    \item $\frac{5x - y}{7} = 1/7(5x - y)$\par
    \textit{The Distributive Property}
    
    \item $5(4 + 7) = 5(7 + 4)$\par
    \textit{The Associative Property of Addition}
    
    \item $(2 + a)b = 2b + ba$\par
    \textit{The Distributive Property}
    
    \item $(-1)(-3 + 4) = (-1)(-3) + (-1)(4)$\par
    \textit{The Distributive Property}
\end{enumerate}


\textit{In Problems 21-27, show that the statements are true by using properties of the real numbers}
\begin{enumerate}[start=21]
    \item $2x(y-7) = 2xy - 14x$\par
    \begin{center}
        \textit{The Distributive Property}\par
        • $2x(y-7)$\par
        • $2xy - 14x$
    \end{center}

    \item $\frac{x}{y}z = x\frac{z}{y}$
    \begin{center}
        \textit{The Commutative Property of Multiplication}\par
        • $\frac{x}{y} z$\par
        • $\frac{xz}{y}$\par
        • $x \frac{z}{y}$
    \end{center}

    \item $(x + y)(2) = 2x + 2y$
    \begin{center}
        \textit{The Distributive Property}\par
        • $(x + y)(2)$\par
        • $2x + 2y$
    \end{center}
    
    \item $a(b + (c + d)) = a((d + b) + c)$
    \begin{center}
        \textit{The Commutative Property of Addition and Associative}\par
        • $a(b + c + d)$\par
        • $a(d + b + c)$\par
        • $a((d + b) + c)$
    \end{center}

    \item $x((2y + 1) + 3) = 2xy + 4x$
    \begin{center}
        \textit{The Commutative Property of Addition and Distributive}\par
        • $x(2y + 1 + 3)$\par
        • $x(2y + 4)$\par
        • $2xy + 4x$
    \end{center}
    
    \item $(1 + a)(b + c) = b + c + ab + ac$
    \begin{center}
        \textit{The Distributive Property}\par
        • $1b + 1c + ab + ac$\par
        • $b + c + ab + ac$
    \end{center}

    \item Show that $(x - y + z)w = xw - yw + zw$.\par
    [\textit{Hint: $b + c + d = (b + c) + d$}]\par
    Simplify the following if possible
    \begin{center}
        \textit{The Distributive Property}\par
        • $xw - yw + zw$
    \end{center}
\end{enumerate}

\end{document}